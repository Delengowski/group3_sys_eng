\documentclass[lettersize,journal]{IEEEtran}
\usepackage{biblatex}
\usepackage{amsmath,amsfonts}
\usepackage{algorithmicx}
\usepackage{algorithm}
\usepackage{algpseudocode}
\usepackage{array}
\usepackage[caption=false,font=normalsize,labelfont=sf,textfont=sf]{subfig}
\usepackage{textcomp}
\usepackage{stfloats}
\usepackage{url}
%\usepackage{dblfloatfix}
\usepackage{verbatim}
\usepackage{graphicx}
\usepackage{hyperref}
\hyphenation{op-tical net-works semi-conduc-tor IEEE-Xplore}
% updated with editorial comments 8/9/2021
\addbibresource{main_paper.bib}
\begin{document}

\title{Extended Resource Conflict Checking and Resolution Controller
Design for Cross-Organization Emergency Response Processes}

\author{Matt Delengowski}

% The paper headers
\markboth{Rowan University Discrete Event Systems (ECE09568) Fall 2025}%
{Shell \MakeLowercase{\textit{et al.}}: A Sample Article Using
IEEEtran.cls for IEEE Journals}

\maketitle

\begin{abstract}
  Discrete Event Systems (DES) models of dynamical systems which
  change in discrete points of time rather than continuously in time,
  and often asynchronously. Examples of DES are in manufacturing,
  logistics, healthcare, and service operations.
  This paper part of a final project requirement for a DES course.
  The requirement for the final project is to pick a subject from DES
  and a related IEEE paper, which was written in the last 5 years,
  and is from a journal that has impact factor $> 5.0$.
  The topic of this paper is attempting to extend the work of
  \cite{main} which uses Petri Nets (PN) to model emergency response
  systems to optimize the allocation and dispersion of resources from
  first responders and down stream to hospitals.
  The extension is recreate the work of \cite{main} and then attempt
  to make it even more efficient.
\end{abstract}

\begin{IEEEkeywords}
  Cross-organization emergency response
  processes, performance evaluation, Petri nets, resolution
  controller design, resource conflict checking.
\end{IEEEkeywords}

\section{Introduction}
\IEEEPARstart{E}{mergency} response systems are highly complex with
many moving parts and stages that
that if performed poorly can have disastrous effects to those
effected by the emergency. \cite{main} uses
a special type of Extended Petri Net to model that flow of not only
information but resources between the
various elements of an Emergency Response System (ERS). This work
takes the invariant reduction algorithms
provided in \cite{main} for simplifying a PN and decises a program to
automatically perform reductions given
a simple text defintion of the PN. \cite{main} takes the traditional
4 tuple PN definition
and extends it to a total 7 tuple definition as shown in equation \ref{eq:PN}.

\begin{equation}
  \Sigma_{\text{CE}} = (P,T,F,\alpha,\beta,W,M_0) \label{eq:PN}
\end{equation}

A subscript of $L$ denotes a logical condition for a transition to
fire, a subscript of $R$ denotes a resource
that must be available for a transition to fire. In other words part
of the work of \cite{main} is to have a PN
that not only has conditional logic of firing off transitions but
additional available resource constraints that
must be met also. To begin defining the PN of equation \ref{eq:PN},
the places set $P = P_L \cup P_R$ is the union
of resouce and logical places. Each resouce place has specific
meaning, resusability and initial number. Additionally,
resource places are differentiated as external vs internal.

\begin{figure}[H]
  \centering
  \includegraphics[width=0.75\linewidth]{resources.png}
  \caption{Resource places types \cite{main}}
  \label{fig:resources}
\end{figure}

The transitions or \textit{acitivites} are denoted by the set $T$ for
which there are 28 elements in total. Activities
can include but are not limited to \textit{Alarm receipt},
\textit{Rescue the wounded}, \textit{Evacuation}, etc. $F$ denotes
the token/resource flow where $F = F_L \cup F_R$. The token flow is
$F_L = (P_L \times T) \cup (T \times P_L)$ and resource flow
$F_R = (P_R \times T) \cup (T \times P_R)$. The weight function $W$
is modified to account for both token flow and resource movement.
$W(f) = 1$ if $f \in F_L$, $W(f) = \#_{\text{req}}(t,p_r)$ is the
required resources for a tranisiton to fire and $W(f) = \#_{\text{sent}}(t,p_r)$
are the resources released when a transition finishes. The intial
moditions $M_0$ are initial tokens defined as  $M_0(p) = 1$ if $p \in P_L$
and initial resources $M_0(p) = \#_{\text{num}}(p)$ if $p \in P_R$.
Lastly, the new additionals to \ref{req:PN} are $\alpha$ and $\beta$
which respectively
define the minimum amount of time an activitiy and the maximum amount
of time an acitivty takes to execute, where $\alpha(t) < \beta(t)$.

\begin{figure}[H]
  \centering
  \includegraphics[width=0.75\linewidth]{disjoint_nets.png}
  \caption{Sub Petri Nets for each ERS component \cite{main}}
  \label{fig:disjoint_pn}
\end{figure}

With the PN defintion for a single component of an ERS defined,
\cite{main} provides an example individual PNs for each component as
defined in \ref{eq:PN}.
Figure \ref{fig:disjoint_pn}. These indivudal PNs are combined by
unioning the respective sets of each component PN to form
figure \ref{fig:whole_pn}. The overall equation governing the super
PN is defined in equation \ref{eq:super_pn} where $i$ is a
positive integer that represents a sub component of ERS.

\begin{figure}[H]
  \centering
  \includegraphics[width=0.75\linewidth]{whole_pn.png}
  \caption{Super Petri Net for ERS \cite{main}}
  \label{fig:whole_pn}
\end{figure}

\begin{equation}
  \Sigma_{\text{CE}_i} = (P_i,T_i,F_i,\alpha_i,\beta_i,W_i,M_{0_i})
  \label{eq:super_pn}
\end{equation}

\section{Invariant Reduction}

The invariant reduction as defined in \cite{main} is a cruicial step
for reducing the verbosity of the
super PN in figure \ref{fig:whole_pn}. To perform the invariant
reduction \cite{main} defines two primary forms
reduction which will be referred to as sequential and fanned
reduction. The sequential reduction is displayed in
figure \ref{fig:sequential_reduction}.a specifically which allows two
sequential activities to be combined if and only
if neither activity consumes or release resources then they can be
safely combined. The act of combining causes the new
transition to take on the sum of individual $\alpha$ and $\beta$ to
create the new minimum and maximum execution times.

Figure \ref{fig:sequential_reduction}.b and
\ref{fig:sequential_reduction}.c are special cases of figure
\ref{fig:sequential_reduction}.a.
Respectively, they are that the first activity can consume resources
and no subsequent activity can release resources or
the final activity can release resources but no activity prior in the
sequence can consume resources. The minimum and maximum
execution times are summed uniformly.

\begin{figure}[H]
  \centering
  \includegraphics[width=0.75\linewidth]{serial_reductions.png}
  \caption{Sequential reduction of activities \cite{main}}
  \label{fig:sequential_reduction}
\end{figure}

The fanned reduction is described in figure
\ref{fig:fanned_reduction} which states that two or more activties
that transition out of a single activity and transition into a single
activity can be combined if none of those
activities consume or release resources. To combine the minimum and
maximum execution times, $\alpha$ and $\beta$,
take the max of each respectively, for each activity being combined.

\begin{figure}[H]
  \centering
  \includegraphics[width=0.75\linewidth]{fanned_reductions.png}
  \caption{Fanned reduction of activities \cite{main}}
  \label{fig:fanned_reduction}
\end{figure}

\section{Applied Reductions}

The invariant reduction rules were applied in \cite{main} to reduce
the number of activities down from 28
to 20. This section will describe in details the individual
reductions applied. Starting with activity $t_{11}$,
figure \ref{fig:1_1_application} shows the application of reduction
1-1 to activities $t_{11}, t_{12}, t_{13}$.
Reduction 1-1 can be utilized because $t_{11}$ alone consumes
resources but it nor any of the subsequent activities
release resources.

\begin{figure}[H]
  \centering
  \includegraphics[width=0.75\linewidth]{rule11_application.png}
  \caption{Application of rule 1-1 \cite{main}}
  \label{fig:1_1_application}
\end{figure}

With the first reduction applied the current number of activites has
been reduced from $28 - 3 + 1 = 26$ activities.
The next application is two reductions applied iteratively. Figure
\ref{fig:2_to_1_2_application} shows two reductions
being applied sequentially. First reduction 2, the fanned reduction,
is applied to $t_{25}, t_{26}$, boxed in red to
create $t_{25,26}$. $t_{25,26}$ can then be reduced by through
$t_{24}$ to $t_{28}$ by utilizing
reduction 1-2. Of the 4 activities, $t_{24}, t_{25,26}, t_{27},
t_{28}$ no activity consumes resources
and only the final activity, $t_{28}$ releases resources.

\begin{figure}[H]
  \centering
  \includegraphics[width=0.75\linewidth]{rule2_rule12_application.png}
  \caption{Application of rule 2 -\> rule 1-2 \cite{main}}
  \label{fig:2_to_1_2_application}
\end{figure}

The current number of activities now being reduced to $26 - 5 + 1 =
22$. The final invariant reduction
application is another sequential application of multiple rules.
Figure \ref{fig:2_to_1_application} shows
reduction 2 being appplied and then followed up reduction 1.
Reduction 2 is depicted in the red circle, which
reduces the fanned activties of $t_{20}, t_{21}$ to create
$t_{20,21}$. $t_{20,21}$ can then be combined with $t_{19}$
through a sequential reduction by applying reduction 1 to create
$t_{19,20,21}$. After this final reduction the overall
number of activites has been reduced to $22 - 3 + 1 = 20$ activities
- a total reduction of 8 activities.

\begin{figure}[H]
  \centering
  \includegraphics[width=0.75\linewidth]{rule2_rule1_application.png}
  \caption{Application of rule 2 -\> rule 1 \cite{main}}
  \label{fig:2_to_1_application}
\end{figure}

\section{Implementation and Discussion}

The novelty of the work presented in the paper starts here. It was
decided to implement a computer program to automatically apply the invariant
reductions when given the text definition of resources for a PN
described by \ref{eq:super_pn}. Figure \ref{fig:pn_def} shows the
table of resources
definitions for the petri net. The table was transcribed to text
format as a \textit{.tsv} (tab separated values). A tsv was chosen
because \cite{main}
uses definitions that include comma characters and to keep the text
as close to source as possible a tab delimiter had to be chosen. The
tsv in question
can be found in the repository linked in the appendix in location
\textit{data/tsvs/table1.tsv}.  A custom parser had to be written to
transform the text
format into multisets used in the table. The custom loader can be
found in \textit{data\/loader.py} which contains a function
\textit{load\_table1()} that
performs the various Extract Load Transform (ELT) operations required
to get the data into a usable format.

\begin{figure}[H]
  \centering
  \includegraphics[width=0.75\linewidth]{pn_def.png}
  \caption{Resource and execution time definition for petri net. \cite{main}}
  \label{fig:pn_def}
\end{figure}

The overall ELT operation is performed by
\textit{src/objects/load\_activities()} which provides a sequence of
\textit{Activity} objects given the \textit{.tsv}
file. The \textit{Activity} class encapsulates all information given
a row from the table presented in \ref{fig:pn_def}. To construct the
\textit{Activity} class,
Python dataclasses were utilized to provide fine tuned knob turning
to reduce boiler plate in coding and provide a faster implementation.

To test the implementation of the Invariant Reduction rules a
dedicated testing library called \textit{pytest} was utilzied.
\textit{pytest} was favored due to
large familiarity with it and the fact that it allows easy test setup
and tear down, as well as assertions to verifying the conditions of
the test. Lastly, \textit{pytest}
provides very convenient debugging and reporting tools of test
results to allow for fast iteration. Figures \ref{fig:pytest_cli} and
\ref{fig:pytest_debug} show the use
of pytest from command line for reporting and as a debugger, respectively.

\begin{figure}[H]
  \centering
  \includegraphics[width=0.75\linewidth]{pytest_results.png}
  \caption{Command line results of pytest with max verbosity.}
  \label{fig:pytest_cli}
\end{figure}

\begin{figure}[H]
  \centering
  \includegraphics[width=0.75\linewidth]{inside_pytest_test.png}
  \caption{Debugger use of pytest}
  \label{fig:pytest_debug}
\end{figure}

The actual implementation of the Invariant Reductions can be found in
\textit{src/invariant\_reductions.py}. Each reduction variant
can be found as an individual function aptly named \textit{rule1()},
\textit{rule1\_1()}, \textit{rule1\_2()}, and \textit{rule2()}.
The application of these reductions had to be wrapped into a singular
overall function call which takes place in \textit{apply\_rules()}.
\textit{apply\_rules()} operates as such, apply all variants of rule
1, and determine which variant, if at all should be applied to various
subsets of activities. The tie breaker logic is of various importance
as situations can arise where 2 variants of rule 1 compete or compete
with different, overlapping, subsets of activities.

For example, activites 11 and 12 could be combined by rule 1-1 or
activities 12 and 13 could be combined by rule 1. How do you choose
what goes first?
Additionally, what about the case from the paper, where rule 1-1 is
applied to 11, 12, and 13 at once? A special function to break ties
was formulated
specifically for breaking ties with variants of rule 1. Once rule 1
is applied then checks for rule 2 take place. If at any time a
reduction takes place
then a flag is set, indicating so, and that the iterative application
of reductions should continue, as described, until no more reductons
take place. Algorithm
\ref{alg:activity_reduction} shows full details in pseudo code.

\begin{algorithm}
  \caption{Iterative Rule-Based Activity Reduction}
  \label{alg:activity_reduction}
  \begin{algorithmic}[1]
    \Require Activity set $T$
    \Ensure Optimized activity set $T_{\text{opt}}$

    \State $T_{\text{opt}} \gets T$
    \State $\mathcal{R}_1 \gets \{r_1, r_{1.1}, r_{1.2}\}$
    \State $\textit{hasReductions} \gets \text{true}$

    \While{$\textit{hasReductions}$}
    \State $C_1 \gets \emptyset$

    \ForAll{$r \in \mathcal{R}_1$}
    \State $R \gets r(T_{\text{opt}})$
    \If{$R \neq \emptyset$}
    \State Add normalized $R$ to $C_1$
    \Else
    \State Add null candidate to $C_1$
    \EndIf
    \EndFor

    \State $W \gets \textsc{TieBreaker}(C_1)$

    \If{$W$ contains a valid reduction}
    \State $T_{\text{opt}} \gets \textsc{Consolidate}(W, T_{\text{opt}}, 1)$
    \EndIf

    \State $R_2 \gets r_2(T_{\text{opt}})$

    \If{$R_2 \neq \emptyset$}
    \State $T_{\text{opt}} \gets \textsc{Consolidate}(R_2, T_{\text{opt}}, 2)$
    \EndIf

    \If{$W$ is null \textbf{and} $R_2 = \emptyset$}
    \State $\textit{hasReductions} \gets \text{false}$
    \EndIf
    \EndWhile

    \State \Return $T_{\text{opt}}$
  \end{algorithmic}
\end{algorithm}

The algorithm as proposed has a time complexity of $\mathcal{O}(n)$
where $n = |T|$ denotes the number of activities in
the input set. Since each successful consolidation strictly reduces
the size of the activity set or decreases a monotonic
measure, the number of iterations is $\mathcal{O}(n)$ in the worst case.

\section{Further Work}
The remaining work of \cite{main} could be written into a computer
program, particularly the algorithm for finding resource conflicts.
An issue was observed when attempting to implement the resource
conflict algorithm. The results could not be matched, starting with the
first sub algorithm. \cite{main} states that by applying the
defintion of figure \ref{fig:alg1_def} the following activities should be
in potential conflict $\{(t_2, t_{10}), (t_4,t_5), (t_5,t_6),
(t_{18},t_{23}), (t_{17},t_{22})\}$. However, by the definition given, it
would papear that $t_2$ and $t_{17}$ should also be in conflict as
both have a required resource set of $\{p_{r2}\}$.

\begin{figure}[H]
  \centering
  \includegraphics[width=0.75\linewidth]{alg1_def.png}
  \caption{Definition of algorithm 1 for resource conflict detection.
  \cite{main}}
  \label{fig:alg1_def}
\end{figure}

Figure \ref{fig:alg1_issues} shows the contradicting results. There
is no clear reason why $t_{2}$ and $t_{10}$ are in conflict but not
not $t_2$ and $t_{17}$. Activities 2 and 17 both require the same
resources exactly. To more specificly define what $P_R$ is,
$P_R = P_{ERR} \cup P_{ECR} \cup P_{IR}$ where $P_{ERR}$ is external
resuable resources, $P_{ECR}$ is external consumable resources,
and $P_{IR}$ is internal resources. The resoureces are defined in
figure \ref{fig:resources}. Again so activity 2, 10, and 17 all require
the same external resusable resource of $p_{r2}$, so why do they all
not conflict? There can only be of one of $p_{r2}$ at any time. I can
only surmise that there is some issue with the paper or a
misunderstanding of the text, probably the latter. It cannot even be
said that perhaps
only two activities can ever be in conflict at once but that would be
contradicted by the fact that activty 4 conflicts with 5 and 5 conflicts with
activity 6 with 4, 5, and 6 all requiring the same resources. With
that said, the author does state that certain places, such as
$p_{r2}$ are listed twice in the PN for clarity but perhaps that caused issues
and there are slight corrections needed in the paper.

\begin{figure}[H]
  \centering
  \includegraphics[width=0.75\linewidth]{alg1_issues.png}
  \caption{Conflicting results of algorithm 1. Why are activities 2
  and 17 not in conflict? \cite{main}}
  \label{fig:alg1_issues}
\end{figure}

\section{Conclusion}
In conclusion an algorithm was successfully devised to implement
invariant reductions for a ERS petri net, given the text defintion'
of its resources places. In short, this can provide use for large
petri nets to provide faster reduction in complexity of the PN and
allow quicker time to conflict indentification and resolution.
Further work could be performed to write computer programs to implement
the conflict identification.

{\appendix[Source Code]
  The source code for conducting experiments and this paper can be
  found at a
  \href{https://github.com/Delengowski/discrete_event_systems_F2025}{public
  github repository.}
  \url{https://github.com/Delengowski/discrete_event_systems_F2025}
}

\printbibliography

\end{document}
